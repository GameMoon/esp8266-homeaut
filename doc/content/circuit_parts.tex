\section{Kapcsolási rajz}

\subsection{ESP8266 programozó}
Az mikrokontrollernek három különböző boot módja van. Az adott mód kiválasztása 3 db láb megfelelő állapotba állításával lehetséges. A panelon egy 6 érintkezős csatlakozón keresztül lehet UART segítségével programozni. A programozó ezenkívül egy tranzisztor segítségével tudja automatikusan a kód feltöltése során programozható állapotba állítani az eszközt. A kapcsolási rajzon tovább látható, hogy bizonyos lábakat fel, illetve lekell húzni a megfelelő működéshez. Későbbiekben a használható periféria kombinációknál ezeket figyelembe kell venni, mert ezzel is szűkül az elérhető lábak felhasználhatósága.

\begin{figure}[!ht]
    \centering
    \includegraphics[width=130mm, keepaspectratio]{figures/programmer.png}
    \caption{ESP8266 programozó bekötése}
    \label{fig:TeXstudio}
\end{figure}

\begin{table}[ht]
	\footnotesize
	\centering
	\begin{tabular}{ | c | c | c | c | c |}
		\toprule
		GPIO15 & GPIO0 & GPIO2 & mód & leírás \\
		\midrule
        Alacsony & Alacsony & Magas & UART & Kód letöltése UART-ról \\
        \hline
        Alacsony & Magas & Magas & Flash & Boot SPI Flash-ből \\
        \hline
		Magas & x  & x & SDIO & Boot SD kártyáról \\
		\bottomrule
	\end{tabular}
	\caption{ESP8266 boot módok}
	\label{tab:TabularExample}
\end{table}
Az ESP lehetőséget biztosít, hogy vezetéknélkül is lehessen frissíteni a firmware-t OTA ( Over The Air) használatával. Ebben az esetben szükséges egy webszervert is használni, illetve figyelembe kell venni a különböző biztonsági kockázatokat.


\subsection{I2C}
Az ESP8266-nak alapértelmezetten egy I2C interfésze van, ami szoftveresen van megvalósítva. Erre az interfészre kapcsolódik a panelon elhelyezkedő kijelző, hőmérséklet szenzor, illetve shift regiszter. Lehetőség van ezenkívül további perifériák egyszerű csatlakoztatására is, mivel helyet kapott külön kivezetés tüskesorra is.

A panelon elhelyezett hőmérséklet szenzor egy LM75-ös digitális hőmérő, aminek SOP8 tokozása van. Számos más hőmérséklet, illetve párataratalom mérő szenzor is használható a megegyező lábkiosztásnak köszönhetően.
\begin{figure}[!ht]
    \centering
    \includegraphics[width=130mm, keepaspectratio]{figures/i2c_devices.png}
    \caption{Hőmérséklet szenzor, illetve I2C kivezetések}
    \label{fig:i2c}
\end{figure}
\clearpage
% alap lábbekötések
% A modul programozása egy 6 pines csatlakozón (Rx,Tx,RTS,DTS,5V,GND) keresztül UART segítségével történik.
\subsection{FET meghajtó áramkörök}
A mikrokontroller lábai maximálisan 12 mA-al terhelhetőek, ezért szükséges különböző meghajtó áramkörök használata. A modulon helyett kapott 2 db P-csatornás, 4 db N-csatornás MOSFET kimeneti meghajtó, illetve elérhető 2db N csatornás MOSFET-el megvalósított bemenet is.

Az N csatornás kimenetek esetében lehetőség van kiválasztani a kapcsolt feszültséget egy forrszem segítségével. Ez lehet 5V vagy a bemeneti tápfeszültség, ami maximálisan 12V. Továbbá külön beállítható ezeknél a kimeneteknél az alapértelmezett állapot a megfelelő fel vagy lehúzó ellenállás beforrasztásával. Az védelem érdekeben ezenkívül belett tervezve egy dióda induktív terhelés esetére.

A P csatornás meghajtónál a kapcsolás egy másik N csatornás MOSFET segítégével történik. Az alap állapotban az kapcsolást végző MOSFET kikapcsolt állapotban van ezt a bemenetére kötött lehúzó ellenállás biztosítja. A P csatornás MOSFET alapállapotának beállítására is szintén lett kialakítva lehetőség.

\begin{figure}[!ht]
    \centering
    \includegraphics[width=140mm, keepaspectratio]{figures/n_p_driver.png}
    \caption{N és P csatornás MOSFET meghajtók}
    \label{fig:p_channel_mosfet}
\end{figure}

\begin{figure}[!ht]
    \centering
    \includegraphics[width=55mm, keepaspectratio]{figures/n_input.png}
    \caption{N csatornás MOSFET bemenet}
    \label{fig:n_channel_mosfet}
\end{figure}







\subsection{Enkóder}
Lehetőség van egy kapcsolóval egybeépített mechanikus enkóder bekötésére a panelon. Ennek a fő szerepe a kijelző által biztosított menürendszerben való navigálás. A kijelző hiánya esetében is több funkció rendelhető hozzá például a különböző kimenetek szabályzása. A beépített nyomógomb egy interrupt segítségével például beállítható, hogy visszakapcsoljon alvás üzemmódból a mikrokontroller.
\subsection{Optokapu}
2 db optokapunak lett hely kialakítva a panelon, amik például P-csatornás MOSFET meghajtóval kombinálva izolált vezérlést tesz lehetőve.

\begin{figure}[!ht]
    \centering
    \includegraphics[width=100mm, keepaspectratio]{figures/opto.png}
    \caption{Optokapu}
    \label{fig:opto_gate}
\end{figure}
\clearpage

\subsection{Shift regiszterek}
A mikrokontrollernek mivel viszonylag kevés szabad GPIO lába áll rendelkezésre, ezért lehetőség van a különböző shift regiszterek használatával növelni az elérhető bemenetek és kimenetek számát, ha az alkalmazáshoz szükséges. A panelon 2db 8 bit-es SPI interfészel, illetve egy 16 bit-es I2C-vel rendelkező shift regiszter helyezhető el. A 8 bitesek egyike párhuzamos bemeneteket alakít sorossá. A másik pedig soros bemenetet alakít át párhuzamos kimenetekké. 
\begin{figure}[!ht]
    \centering
    \includegraphics[width=140mm, keepaspectratio]{figures/8bit_shift_registers.png}
    \caption{8bites shift regiszterek kapcsolási rajza}
    \label{fig:shift_register}
\end{figure}
A 16 bites shift regiszteren külön konfigurálható, hogy milyen állapotban legyenek a lábai. Egy plusz funkció, hogy interrupt rendelhető bármelyik bemeneti módban lévő lábhoz. A felhasznált terület csökkentése érdekében, illetve mivel nem minden alkalmazáshoz szükséges plusz bemenet vagy kimenet, ezért a 16bites és a 8bites IC-k ki és bemenetei ugyanazokra a lábakra vannak kivezetve.
\begin{figure}[!ht]
    \centering
    \includegraphics[width=95mm, keepaspectratio]{figures/16bit_shift_register.png}
    \caption{16bites shift regiszter kapcsolási rajza}
    \label{fig:shift_register}
\end{figure}


\subsection{Infra adó/vevő}
Számos otthoni eszköz használ különböző infrás megoldásokat főleg a távoli vezérléshez. Ezeknek az eszközöknek az illesztése érdekeben a panelen helyett kapott egy infra adó és egy vevő, amit a mikrokontroller erre szánt lábaira bekötve létrejöhet a kommunikáció. Lehetőség van ezáltal a már meglévő távirányítókkal a modul irányítása is, így például elhagyható az enkóder használata.
A infra led meghajtása egy N csatornás MOSFET-el történik, mivel nagyobb áram szükséges a LED meghajtásához, mint a mikorokontroller lába által biztosított 12 mA.
\begin{figure}[!ht]
    \centering
    \includegraphics[width=130mm, keepaspectratio]{figures/infra.png}
    \caption{Infra adó/vevő}
    \label{fig:ir_ado_vevo}
\end{figure}


\subsection{Kijelzők, LED-ek}
Opcionálisan elhelyezhető a panelon egy I2C interfésszel rendelkező 128x64-es felbontású OLED kijelző. A kijelzőn könnyen megjeleníthetőek az eszközhöz kapcsolt különböző szenzorok értékei minimális kód hozzáadásával. Megjeleníthetők továbbá alapvető információk pl. hálozat állapota, GPIO pinek állapota is, amik segítségül szolgálhatnak az adott alkalmazás fejlesztésénél. A kijelző modulnak eltérő lábkiosztású modelljei is léteznek, ezért két forrszem segítségével könnyedén módosítható az éppen használt típusnak megfelelően.
\begin{figure}[!ht]
    \centering
    \includegraphics[width=120mm, keepaspectratio]{figures/display.png}
    \caption{OLED kijelző bekötése}
    \label{fig:oled}
\end{figure}

Elérhető továbbá egy normál, illetve egy RGB led, amiket szintén sokféle képpen lehet felhasználni. A kijelző elhagyása esetén lehet állapot jelzőnek használni. A panelre nem került külön led a tápfeszültség meglétének jelzésére, de a ledhez tartozó mindkét forrszem beforrasztásával ez is megoldható. Az RGB led vezérléséhez 5 V-os jelszint szükséges. A kapcsoláson jól látható, hogy egy "és" kaput használva történik a jelszint átalakítás. Az "és" kapu a bemenetére kötött 3,3 V-ot már logikai 1-esnek érzékeli, illetve a bemenetei összevannak kötve, ezért a kimenetet éppen a bejövő jelnek megfelelően kapcsolja.
\begin{figure}[!ht]
    \centering
    \includegraphics[width=150mm, keepaspectratio]{figures/rgb_led.png}
    \caption{RGB Led}
    \label{fig:power_supply}
\end{figure}
Egyszerű számok, illetve időpontok kijelzésére elegendő lehet csak néhány 7 szegmenses kijelző. Ezt figyelembe véve ki lett alakítva egy csatlakozástatási pont, ahova egy ilyen kijelzőket vezérlő modul köthető be. Ilyen például a TM1637 típusú 7 szegmenses vezérlő meghajtó.
\begin{figure}[!ht]
    \centering
    \includegraphics[width=150mm, keepaspectratio]{figures/led_connector.png}
    \caption{Led, 7 szegmenses kijelző csatlakoztatási pont}
    \label{fig:led_7segment}
\end{figure}



\subsection{ADC}
Az ESP8266 mikrokontrollernek csak egy darab lába képes analóg bemenetet olvasni. Az átalakítást egy 10 bites ADC végzi, aminek a bemeneti tartománya 0-10V-ig terjed. A bemenet átskálázását egy feszültségosztóval állítható be. Akkumlátoros tápellátás esetén például használható a bemeneti tápfeszültség mérésére.

\clearpage
\subsection{Tápellátás}
A megvalósított eszköz tápellátására több lehetőség is biztosított. Egyrész található rajta egy szabványos tápcsatlakozó, ami rávan kötve egy 5V-os feszültségszabályozóra. Az alkatrészek méretezése és a szabályzó IC miatt a bemenetre 6.5V - 12V feszültség csatlakoztatható. A másik lehetőség egy micro USB-n keresztül 5V-os bemenet, illetve a programozón keresztül is lehetséges az áramkör táplálása. A bejövő feszültséget két darab lineáris feszültségszabályzó alakítja át. Az első 5V-ra majd ennek a kimenete rá van kötve egy 3,3V-ot előállítóra. A stabil feszültségek biztosítása érdekében szűrő kondenzátorok kerültek az IC-k ki és bemeneteire is. A további kiegészítő áramkörök egyszerű tápellátása érdekében a különböző tápfeszültségek több ponton is kilettek vezetve a NYÁK-on.

\begin{figure}[!ht]
    \centering
    \includegraphics[width=130mm, keepaspectratio]{figures/power.png}
    \caption{Tápellátás}
    \label{fig:TeXstudio}
\end{figure}


