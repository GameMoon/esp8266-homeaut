\pagenumbering{roman}
\setcounter{page}{1}

\selecthungarian

%----------------------------------------------------------------------------
% Abstract in Hungarian
%----------------------------------------------------------------------------
\chapter*{Kivonat}\addcontentsline{toc}{chapter}{Kivonat}

% Jelen dokumentum egy diplomaterv sablon, amely formai keretet ad a BME Villamosmérnöki és Informatikai Karán végző hallgatók által elkészítendő szakdolgozatnak és diplomatervnek. A sablon használata opcionális. Ez a sablon \LaTeX~alapú, a \emph{TeXLive} \TeX-implementációval és a PDF-\LaTeX~fordítóval működőképes.

Egyre több átlagos háztartásban terjednek el a különböző okos eszközök, amiknek a célja általában plusz kényelmi funkciók biztosítása. Számos nagyobb cégnek van már olyan megoldása, hogy az általa gyártott eszközök segítségével egy teljes lakásautomatizálási rendszer kiépíthetővé válik. Ezeknek a rendszereknek a legnagyobb hátrányuk általában a más gyártók termékeivel való bonyolúlt integráció. Emellett azt is figyelembe kell venni, hogy még mindig elég költséges lehet egy ilyen rendszer kiépítése.

A dolgozatom során egy ESP8266 alapú mikrokontroller köré tervezett okosotthon rendszert szeretnék bemutatni. Tervezési alapelvnek számított, hogy a már meglévő eszközökkel kompatbilis legyen. Fontos szempont volt továbbá, egy univerzálisan felhasználható áramkör létrehozása, amibe elegendő legyen csak az adott feladathoz szükséges alkatrészek beépítése, ezzel is csökkentve a kiadásokat.
A megvalósított rendszerben lévő egységek továbbá képesek automatikusan létrehozni és menedzselni a közöttük lévő hálózatot. Ennek alapját a mikrokontroller gyártója által megalkotott ESPNOW protokoll adja.

\vfill
\selectenglish


%----------------------------------------------------------------------------
% Abstract in English
%----------------------------------------------------------------------------
\chapter*{Abstract}\addcontentsline{toc}{chapter}{Abstract}

More and more ordinary households are spreading various smart devices, which are generally designed to provide extra convenience features. Many larger companies already have the solution to build a complete home automation system using the devices they produce. The biggest disadvantage of these systems is usually the complex integration with other manufacturers' products. It should also be borne in mind that it can still be quite expensive to build such a system.

In my thesis I would like to introduce a smarthome system designed around an ESP8266 based microcontroller. It was a design principle to be compatible with existing devices. Another important consideration was the creation of a universally usable circuit, which would be sufficient to solder only the parts necessary for the task, thereby reducing the costs.
The units in the implemented system are also capable of automatically creating and managing the network between each other. It is based on the ESPNOW protocol developed by the microcontroller manufacturer.


\vfill
\selectthesislanguage

\newcounter{romanPage}
\setcounter{romanPage}{\value{page}}
\stepcounter{romanPage}