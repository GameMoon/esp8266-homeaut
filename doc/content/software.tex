\section{Szoftver}

A szoftver az áramkörhöz hasonlóan többi opcionálisan használható elemből épül fel. A felhasznalónak lehetősége egy konfigurációs fájlban megadnia, hogy mely elemeket szeretné használni. Ilyen beállítható érték például az ESPNOW hálózat, MQTT használata, de ide tartozik a kijelző és az enkóder jelenléte is. Lehetőség van már meglévő szoftvereket is használni, mint például a ESPHome-t. Ez szintén egy konfig fájl alapján állítja elő az eszközre feltölthető kódot.

\subsection{Üzenetek}
Az alapértelmezett kommunikáció az egység közök ESPNOW használatával történik. A protokoll biztosít az átküldendő keretekben egy 0-250 byte méretű payload-ot, ahova a felhasználó tudja az átküldendő adatot elhelyezni. Az általam definiált csomag tartalmazza mindig a küldő és fogadó fel azonosítóját, hogy köztes egységek tudják hogyan kell továbbítani, illetve egy parancsot. Ezt követhet egy adat mező, ami a parancsnak megfelelő értékeket tartalmazza.

\begin{table}[ht]
	\footnotesize
	\centering
	\begin{tabular}{ | c | c | c | c |}
		\toprule
		Küldő MAC címe & Cél MAC címe & Parancs & Adat \\
		\midrule
        8 byte & 8 byte & 6 byte & 0 - 233 byte \\
	\end{tabular}
	\caption{Üzenet felépítése}
	\label{tab:TabularExample}
\end{table}

A rendszerben jelenleg az alapvetően a hálozat felépítéséhez szükséges üzenetek vannak leimplementálva. Szükség esetén azonban könnyen lehet újat hozzáadni. Az elérhető parancsok listája.

\subsubsection{REQUEST\_PARENT}
A szülővel nem rendelkező egységek ezt az üzenetet küldik el broadcast címzett megadásával, hogy kihez tudnak éppen csatlakozni
\subsubsection{ACCEPT\_CHILD}
A szülő módban lévő egységek ezzel tudnak válaszolni csatlakozni kívánó eszközöknek, hogy csatlakozhatnak hozzá.
\subsubsection{ACCEPT\_PARENT}
A gyerek egység válasza az ACCEPT\_CHILD üzenetre a szülő felé, hogy őt választotta.
\subsubsection{STATUS}
Ezt a parancsot az egyes eszközök periodikusan küldik a hozzájuk kapcsolódott eszközöknek. Az adatmező ebben az esetben az adott egységhez kapcsolt perfiériák (pl. szenzorok) értékeit tartalmazza. 

\begin{table}[ht]
	\footnotesize
	\centering
	\begin{tabular}{ | c | c | c | c |}
		\toprule
		Periféria típus & Perféria azonosító & Érték \\
		\midrule
        1 byte & 1 byte & 4 byte \\
	\end{tabular}
	\caption{Formátum az adatmezőben}
	\label{tab:TabularExample}
\end{table}

\subsubsection{UPDATE}
Ilyen típusú üzenet segítségével lehet, az adott egységhez kapcsolt eszközök értekeit módosítani. Ilyen lehet például a GPIO állapotok.
\subsubsection{TEST}
A hálózat tesztelése érdekében használt csomag, amit az egységek visszaküldenek automatikusan a küldőnek.


\subsection{Hálózat}
A hálózatban két különböző típusú csomópontot lehet megkülönböztetni. Az egyik a gyökér csomópont, illetve a normál csomópont. Mind a két típusú egység rendelkezik egy beállítható szülő flag-el, ami ha igaz értékre van állítva, akkor válaszolhat az újonnan érkező kapcsolódni kívánó egységeknek.

\subsubsection{Gyökér csomópont}
A hálózat felépítése ezek segítségével kezdődik. Ilyen egység akkor keletkezik, ha az adott csomópont többszöri REQUEST\_PARENT üzenet küldése után sem kap választ. Ebben az esetben kinevezi magát kezdő csomópontnak és beállítja szülő flag-et, ezzel is jelezve, hogy hozzá már lehet kapcsolódni. Ezeknek a csomópontoknak olyan további különleges tulajdonságuk is van, hogy ők tudnak kommunikálni a külvilággal, feltéve ha a konfigurációs fájlban engedélyezve van az MQTT használata. Az ide beérkező STATUS üzeneteket átalakítják MQTT-ben használt topikokká majd ezeket publikálják. Az általuk publikát topikokra fel is vannak iratkozva, ezért módosítás esetén ők alakítják vissza a Brooker-től érkező infromációkat UPDATE üzenetekké majd továbbítják hálózatukon.

\subsubsection{Normál csomópont}
A normál csomópont akkor keletkezik, ha egy egység sikeresen rákapcsolódik egy a gyökér vagy egy másik csomópontra, ami szülőként is funkciónal. Sikeres kapcsolódás után engedélyezheti, hogy ő rá is lehessen kapcsolódni.

\begin{figure}[!ht]
    \centering
    \includegraphics[width=150mm, keepaspectratio]{figures/mesh.png}
    \caption{A hálózat felépítése}
    \label{fig:TeXstudio}
\end{figure}

A hálózat felépítése során fontos, hogy egy csomópontra csak akkor lehessen csatlakozni, ha már ő is csatlakoztatva van valakihez ( kivétel a gyökér egység ). Ellenkező esetben létre jöhet olyan hálozat, ami nem tud a külvilággal kommunikálni. A másik fontos tulajdonság, hogy egy egységnek csak egy szülője lehet, hogy ne alaklujanak ki hurkok, ahol az üzenetek végtelenségig keringenének. Az utóbbi esetet egyébként egy plusz TTL paraméter bevezetésével kilehetne küszöbölni. A feltételek betartása esetében a hálózatnak végül egy fa struktúrája lesz.

Egy egység kiesésének detektálás időtúllépéssel detektálható. Minden egység tárol a hozzá kapcsolódott eszközökhöz egy időtúllépési számlálót, amiknek az értékeit periódikusan növeli. STATUS üzenet érkezése esetén az adott számlálót 0-ra állítja. A számlaló ha elér egy megadott értéket, akkor törli az egységet a táblájából. Az olyan esetekben, amikor a törölt egység megegyezik a szülő egységgel, akkor ismét elkezdi a szülő keresési lépést.

\subsection{Szenzorok}