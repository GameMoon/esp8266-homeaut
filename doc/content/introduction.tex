%----------------------------------------------------------------------------
%----------------------------------------------------------------------------
\chapter{\bevezetes}
%----------------------------------------------------------------------------

% A bevezető tartalmazza a diplomaterv-kiírás elemzését, történelmi előzményeit, a feladat indokoltságát (a motiváció leírását), az eddigi megoldásokat, és ennek tükrében a hallgató megoldásának összefoglalását.
% A bevezető szokás szerint a diplomaterv felépítésével záródik, azaz annak rövid leírásával, hogy melyik fejezet mivel foglalkozik.

Az elmúlt években egyre gyakrabban jelennek meg az átlagos felhasználók számára tervezett különböző okos eszközök, amelyek általában plusz kényelmi funkciókat biztosítanak. Ide sorolhatók például az okosizzók, amelyeket bárhonnan tudjuk a lakásból egy telefon segítségével vezérelni, különböző színeket beállítani. Az okos kávéfőzők, amik távolról indíthatók, illetve időzítő beállításával ébredés után már rögtön vár minket az általunk kiválaszott kávé. Az okosotthonban lévő eszközöket továbbá lehet hasznosabb funkciókra is használni, amelyek például a biztonságot vagy a felhasznált energiát csökkentik. Ilyen eszközök a különböző kamerával vagy mozgás érzékelővel rendelkező eszközök, ami tud jelezni a tulajdonosnak esetleges betörés esetén. Kisgyerekek és idősek monitorozására is fontos alkamazás lehet. Az energia felhasználás csökkentése lehetséges például a fűtés decentralizált szabályzásával, vagy az éppen üres szobákban a lámpa lekapcsolása. 

Léteznek nagyobb cégek által készített teljes megoldások is egy lakás felokosítására. Ezeknek a megoldásoknak az egyik nagy hátránya, hogy általában csak a saját termékeik csatlakoztathatók könnyen. Más gyártók eszközei sokszor csak körülményesen integrálhatók a rendszerbe. Továbbá egy másik hátránynak tekinthető, hogy egy teljes konfiguráció kiépítése költséges is lehet. A smarthome rendszereket csoportosíthatjuk az alkotó egységek kapcsolódási módja szerint. Ez alapján vezetékes és vezetéknélküli rendszerekeket különböztethetünk meg. A vezetékes rendszereket általában új építésű házaknál használnak, mivel itt a tervezéskor tudnak a plusz kábelek elvezetetésével kalkulálni. Ezek a rendszerek megbízhatóbbak, gyorsabbak és biztonságosabbak a vezetetéknélküli megoldásnál, mivel fizikailag kell kapcsolódni a hálozathoz a kommunikácóhoz. A szélesebb körben elterjedt rendszerek inkább a vezetnélküli megoldást használják, mivel ezekhez nem kell átépíteni a házat, ezért könnyebben és olcsóban használhatóak. Az összeköttetésüket általában a már meglévő WiFi router segítségével biztosítható. Itt felmerül az egyik legnagyobb problémája az okos eszközönek, hogy fejlesztés során nem mindig a biztonság a fő szempont. A WiFi hálózatok feltörhetősége miatt pedig többször lehetett már hallani, hogy átveszik a hackerek a vezérlést az eszközök felett és ezek segítségével indítanak támadásokat. A másik hátrányuk, hogy sok eszköz kapcsolódása esetén leterhelődik a router és sebesség csökkenés, illetve kommunikációs hibák léphetnek fel.

A dolgozatban megvalósított rendszerben az egységek vezetéknélkül kommunikálnak egymással, mivel fontos szempont volt, hogy egy olcsó és a meglévő termékekkel kompatbilis eszköz jöjjön létre. Az egyik fő különbség, hogy az egységek automatikusan építik a fel a hálózatot és nem feltétlen szükséges WiFi router alkalmazása. Ezzel kilehet küszöbölni, hogy az interneten keresztül hozzáférjenek az eszközeinkhez. Az egységek egymással titkosított üzenetek segítségével kommunikálnak, ezzel is nehezítve a ház közelében lévő támadók feladatát. A másik főszempont volt az áramkör tervezésénél, hogy minél több meglévő rendszerrel kompatibilis legyen. Ezek lehetnek okos eszközök, de lehet például régebbi típusú légkondi vagy házimozi rendszer, amik például infra távirányítóval vezérelhetőek. A NYÁK-on ennek megfelelően elegenedő az adott alkalmazáshoz szükséges eszközök és néhány alap komponens beforrasztása.
A továbbiakban ennek a rendszernek a bemutatásról lesz szó. A bemutatás során kiderül, hogy milyen meglévő rendszerekkel lehet együtt használni, milyen áramköri elemek kaptak helyet a panelon, hogyan is működik pontosan a hálózat felépítés és a kommunikáció és hogy, milyen alkalmazásokra és hogyan lehet fel használni az egységet. 


% Smarthome rendszerek előnyei hátrányi, hol lehetne őket használi
% megfigyelés, energy save
% pl. idősek felügyelete