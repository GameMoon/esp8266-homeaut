
% Szakdolgozat, diplomaterv esetén a dolgozat felépítése valami ehhez hasonló, témától függetlenül lehet természetesen eltérés:
% -Bevezetés, 1-2 oldal, a téma elhelyezése, miről lesz szó
% -Szakirodalmi áttekintés: a dolgozat kb harmada, szükséges előismereteket tartalmazza, illetve szakdolgozat esetén kb 5, diplomaterv esetén kb 10 szakmai cikk hivatkozása, 
%   1-1 bekezdésben összefoglalva, kik, mivel mit csináltak, milyen eredményre jutottak, az aktuális dolgozathoz hogy kapcsolódik, 
%   miért releváns. Kutatós témák esetén ez lényegesen nagyobb hangsúlyt kap (több cikk is hivatkozható!), 
%   fejlesztős feladatok esetén hasonló termékek bemutatása, elmélet amire a tervezés (a későbbiekben bemutatandó munka) épül.
% -Tervezés (témától függően más is lehet a fejezet címe): A saját munka bemutatása. Hardver fejlesztős dolgozat esetén kapcsolási rajz részletek megmagyarázva stb.
% -Értékelés: Szimulációs eredmények, valós mérések bemutatása, a megvalósítás mennyire sikerült, eltérések a várttól.
% -Összefoglalás: a dolgozat összefoglalása, kb 1-2 oldal, jövőbeli tervek.
% A szakirodalmakhoz lehet használni guglit, de ne ez legyen az első. Tanszéki és kolis hálóról elérhetőek online folyóiratok, ez a legjobb forrás (gugli is indexel) pl:
% -IEEE Xplore
% -Elsevier
% -Springerlink
% (-Google scholar)
% -stb
% A wikipedia, youtube és online hírportálok hivatkozását lehetőleg mindenki kerülje. Ha mégis nagyon releváns, ők is hivatkozni szoktak az eredeti forrásra, azt nézzétek meg.
% Szakirodalmak feldolgozása:
% -Absztrakt elolvasása, el kell dönteni, hogy releváns-e a cikk vagy sem
% -Bevezetés átolvasása, általában egy jó áttekintése a témakörnek
% -Ábrák átnézése
% -Conclusion rész elolvasása
% -Aki jobban el akar merülni benne, a többi rész is elolvasható
% -Ezek alapján néhány mondatos összefoglaló generálása

% Hasonló rendszerek
% https://esphome.io/
% https://www.openhab.org/
% Kész termékek
% https://www.itead.cc/sonoff-pow.html


\chapter{Szakirodalmi áttekintés}
\section{ESP8266}
Az Espressif cég által gyártott mikrokontroller, amelyen található egy 32bit-es 80Mhz-es Tensilica processzor, 50kB RAM, egy külső flash chip, ami tárolja a felhasználói programot, illetve különböző perifériák. Emellett tartalmaz egy teljes WiFi stack-et, amely támogatja a 802.11 b/g/n szabványokat. A modulnak két virtuális WiFi interfésze van, amikkel tudja biztosítani, hogy állomásként (station), elérési pontként (softAP) vagy egyszerre mind a két módon üzemeljen. 

\begin{figure}[!ht]
    \centering
    \includegraphics[width=150mm, keepaspectratio]{figures/esp8266funcdiag.png}
    \caption{Az ESP8266 funkcionális diagramja}
    \label{fig:TeXstudio}
\end{figure}

A ESP8266-hoz léteznek különböző típusú moduljai (pl. ESP-01, ESP-12), amelyek általában a panelre kivezetett GPIO számban térnek el főleg. Összesen 17 db GPIO láb áll rendelkezésre, amikhez különböző funkciók rendelhetők.
Az IoT rendszerekben elterjedten használják, ami elsősorban az alacsony árának és a vele kompatibilis számos fejlesztői környezetnek (pl. Arduino, VStudio) is köszönhető. A gyártó alapvetően 3 különböző SDK-t biztosít a fejlesztéshez. Az Arduino core-t, NONOS SDK-t, illetve a RTOS SDK-t, ami FreeRTOS alapú.

Számos versenytársa létezik például a RTL8710, AIR602 amik eltérő processzort használnak, de teljesítményben és fogyasztásban nincs nagy eltérés. Ezenkívűl létezik már az ESP-nek egy újabb verziója az ESP32, ami már 2-magos processzorral, nagyobb memóriával és több perifáriával rendelkezik.

% http://wiki.seeedstudio.com/W600_Module/
% különböző modul típusok
% elérhető lábszám
% alacsony ár
% miben lehet rá fejleszteni
% iotban elterjedt, hasonló ellenfele RTL8710


\section{ESPNOW}
Az ESPNOW egy kommunikációs protokoll, ami WiFi-hez hasonlóan a 802.11-es szabvány alapján működik. Az egyik eltérés a WiFi-hez képest, hogy nincs külön kapcsolat felépítés az eszközök között. Ennek köszönhetően kevesebb az overhead a kommunikáció során, ezzel is gyorsítva azt.

A fizikai és hozzáférési réteg felett a gyártó által definiált egyedi frame található. A frame tartalmaz egy Body mezőt, ami tartalmazhatja a saját alkalmazásunk payload-ját. Ennek a mérete 0-250 byte között változhat. A frame ezenkívűl tartalmaz még különböző a gyártó által használt mezőt, mint pl ESPNOW verziószámát, egyedi azonosítót ami az adott ESP MAC címének első 3 byte-ból áll.

Kommunikációhoz az ESP-kt először összekell párosítani, ehhez szükséges megadni a másik modul MAC címét. Egy modul maximum 20 másik eszközzel lehet összepárosítva. Lehetőség van broadcast üzenet küldésére, ehhez a FF:FF:FF:FF:FF:FF címet is hozzá kell adni a párosított eszközök közé.

% nincs külön kapcsolat két eszköz között
% custom frame format
% sikeres küldés esetén ack, TODO mérni kellene
% https://docs.espressif.com/projects/esp-idf/en/latest/api-reference/network/esp_now.html

\section{MQTT}

\section{NodeRED}
% https://nodered.org/

\section{Home Assistant}
% https://www.home-assistant.io/

\section{ESPHome}
% https://esphome.io/

\section{Openhab}

% https://ieeexplore.ieee.org/document/8603575